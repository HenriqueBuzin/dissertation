% ----------------------------------------------------------
\chapter{Resultados}\label{cap:resultados}
% ----------------------------------------------------------

Esta seção apresenta os valores medidos a partir dos testes realizados na arquitetura proposta. 
Para cada conjunto de valores coletados, são descritos o método de medição e uma análise interpretativa dos resultados, destacando seu significado e discutindo seu impacto no desempenho do sistema. 
Além disso, sempre que aplicável, os resultados são comparados com aqueles obtidos em outras arquiteturas ou trabalhos relacionados, a fim de contextualizar o desempenho e evidenciar os avanços proporcionados pela solução proposta.

% ----------------------------------------------------------
\section{Valores Medidos}
% ----------------------------------------------------------

Os valores apresentados nas Tabelas~\ref{tab:latencias} e~\ref{tab:memoria} foram obtidos a partir de \textit{timestamps} registrados nos pontos de entrada e saída de cada componente, bem como de medições de consumo de memória realizadas durante o envio das mensagens.  

A Tabela~\ref{tab:latencias} mostra as latências mínimas, máximas e médias para cada segmento do fluxo, desde a recepção inicial no medidor até o processamento final na plataforma HPCC Systems.  
Já a Tabela~\ref{tab:memoria} apresenta o consumo médio de memória por componente do sistema, obtido durante a execução contínua de 680.000 requisições, o que garante consistência estatística para os valores apresentados.

\begin{table}[htb]
    \caption{\label{tab:latencias}Latências mínimas, máximas e médias medidas}
    \centering
    \begin{tabular}{|c|c|c|c|}
        \hline
            \textbf{Caminho} & \textbf{Mínimo (ms)} & \textbf{Máximo (ms)} & \textbf{Média (ms)} \\ \hline
            Meter → Primary Node & 456 & 2455 & 1455 \\ \hline
            Primary Node → Fog Nodes & 1 & 3 & 2 \\ \hline
            Fog Nodes → Aggregator & 2 & 3 & 2,5 \\ \hline
            Aggregator → HPCC Systems & 600 & 1600 & 1100 \\ \hline
    \end{tabular}
    \fonte{Do autor.}
\end{table}

A análise das latências evidencia que o maior tempo de transmissão ocorre entre o Agregador e o HPCC Systems, etapa que concentra a maior parte da latência total.

\begin{table}[htb]
    \caption{\label{tab:memoria}Consumo médio de memória por componente}
    \centering
    \begin{tabular}{|l|c|}
        \hline
            \textbf{Componente} & \textbf{Memória (MiB)} \\ \hline
            Load Balancer & 37,37 \\ \hline
            Agregador & 40,14 \\ \hline
            Nó de Névoa (Energia) & 276,60 \\ \hline
            Simulador de Medidor HTTP & 19,97 \\ \hline
    \end{tabular}
    \fonte{Do autor.}
\end{table}

Em relação ao consumo de memória, observa-se que os nós de névoa apresentam maior demanda devido às operações de pré-processamento e armazenamento temporário, enquanto os componentes de controle, como o nó primário de névoa e o nó agregador, consomem menos recursos.

% ----------------------------------------------------------
\section{Discussão dos Resultados}
% ----------------------------------------------------------

Os resultados obtidos demonstram que a arquitetura proposta apresenta um baixo consumo de recursos quando comparada com o estudo de D’Agostino~\textit{et al.}, no estudo relata-se que a execução de um ambiente de névoa implementado em Docker Compose consumia aproximadamente 1,2~GB de memória com apenas um contêiner ativo \cite{dagostino2025}. 
Mesmo com todos os componentes em funcionamento, o sistema proposto manteve o consumo total em cerca de 374~MiB, o que corresponde a apenas 31\% do valor relatado naquele estudo, representando uma redução aproximada de 69\% no uso de memória. 

O sistema foi desenvolvido em conformidade com os principais princípios da \textit{OpenFog Reference Architecture (IEEE Std 1934-2018)}, com ênfase na escalabilidade, abertura, autonomia, programabilidade, e hierarquia \cite{openfog2018}.

A separação funcional entre as camadas de borda, névoa e nuvem segue a organização preconizada pelo modelo de referência, favorecendo o processamento distribuído e a expansão flexível da arquitetura. 
Além disso, a abstração de protocolos contribui para a interoperabilidade entre sistemas heterogêneos, em consonância com os conceitos do OpenFog, além da coordenação entre nós primários implementa uma forma simplificada de tomada de decisão local.

Quando comparado a trabalhos recentes, o sistema proposto complementa e estende abordagens existentes. 
Rahman e Hussain~\cite{rahman2024} exploraram a interoperabilidade semântica em um \textit{testbed} com múltiplos domínios de névoa, enquanto a arquitetura aqui apresentada enfatiza a interoperabilidade de protocolos e a composição modular. 
Oliveira~\textit{et al.}~\cite{oliveira2024} aprimoraram o posicionamento modular de aplicações, mas sem coordenação entre diferentes névoas; nossa solução amplia essa capacidade por meio de redirecionamento adaptativo de requisições entre nós primários. 
Já Boudieb~\textit{et al.}~\cite{boudieb2024} abordaram o balanceamento dinâmico de carga com \textit{deep reinforcement learning}, ao passo que a presente proposta alcança uma distribuição adaptativa mais simples, baseada em coordenação autônoma e que consome menos recursos.
