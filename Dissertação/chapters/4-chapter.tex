% ----------------------------------------------------------
\chapter{Proposta}\label{cap:proposta}
% ----------------------------------------------------------

Este capítulo descreve a arquitetura modular de computação em névoa desenvolvida ao longo deste trabalho. O conteúdo está organizado para apresentar, inicialmente, a visão geral e o papel de cada tipo de nó. Em seguida, são detalhados os componentes, a organização interna dos nós de névoa, os mecanismos de comunicação entre domínios e o fluxo de dados previsto no sistema.

% ----------------------------------------------------------
\section{Visão Geral}
% ----------------------------------------------------------

A arquitetura proposta é composta por três tipos de nós, cada um com responsabilidades distintas no fluxo de dados e na coordenação do processamento distribuído. O nó primário atua como ponto de entrada de um domínio de névoa, gerenciando dispositivos, distribuindo carga e coordenando a comunicação com outros domínios. Os nós de névoa executam o processamento local e oferecem serviços configuráveis para tratamento dos dados. O nó agregador consolida e organiza as informações processadas antes do envio à nuvem.

A Figura~\ref{fig:arquitetura_proposta} apresenta uma visão de alto nível, mostrando o caminho percorrido pelos dados desde os dispositivos na borda até a nuvem, incluindo as interações e responsabilidades de cada tipo de nó.

\begin{figure}[htb]
    \caption{\label{fig:arquitetura_proposta}Visão geral da arquitetura proposta.}
    \begin{center}
        \includegraphics[width=1\linewidth]{images/arquitetura_proposta.png}
    \end{center}
    \fonte{Do autor.}
\end{figure}

% ----------------------------------------------------------
\section{Componentes}
% ----------------------------------------------------------

Esta seção apresenta, em detalhe, as funções e responsabilidades de cada tipo de nó que compõe a arquitetura: nó primário, nós de névoa e nó agregador.

% ----------------------------------------------------------
\subsection{Dispositivos}
% ----------------------------------------------------------

Os dispositivos de borda são responsáveis por coletar dados diretamente do ambiente físico, atuando como o ponto inicial do fluxo de informações dentro da arquitetura de névoa. Eles podem ser sensores, medidores inteligentes, câmeras ou outros equipamentos capazes de capturar dados em tempo real.

Esses dispositivos podem ser fabricados por diferentes empresas e utilizar diversos protocolos de comunicação, o que torna essencial a existência de mecanismos de padronização e compatibilidade nos níveis superiores do sistema. Essa diversidade permite maior flexibilidade na integração de tecnologias, mas também exige que os nós de névoa consigam interpretar corretamente as informações recebidas, independentemente do protocolo utilizado.

Além disso, os dispositivos mantêm comunicação constante com o nó de névoa para atualizar seu estado e garantir que os dados sejam enviados corretamente. Caso percam a conexão, podem armazenar temporariamente as informações até que a comunicação seja restabelecida.

Dessa forma, os dispositivos de borda formam a base do ecossistema de computação em névoa, aproximando a coleta de dados do ambiente físico e contribuindo para a eficiência, interoperabilidade e agilidade de todo o sistema.

% ----------------------------------------------------------
\subsection{Nó Primário}
% ----------------------------------------------------------

O nó primário de névoa é a porta de entrada da arquitetura, ela recebe dados dos dispositivos de borda e distribui a carga de trabalho entre os nós de névoa disponíveis, direcionando as requisições conforme a especialização de cada um.

Quando um novo nó é iniciado, ele se registra no nó principal informando sua especialização, como água ou energia, e envia atualizações periódicas para indicar que continua ativo. Caso pare de enviar esses sinais, é removido automaticamente.

A distribuição da carga é balanceada, seguindo uma abordagem parecida com o Round Robin. Cada nó processa uma requisição por vez, mas sua especialização é respeitada.

Ao ser inicializado, o nó primário avisa os demais nós primários que está ativo e informa quantos nós especializados possui. Durante o funcionamento, ele monitora o número de dispositivos e requisições. Se houver desequilíbrio, verifica com outros nós primários qual tem mais capacidade disponível.

Por exemplo, se houver poucos nós especializados, o nó principal pode redirecionar requisições para outro domínio com mais recursos. Quando detecta sobrecarga, ele envia uma mensagem de broadcast para os outros nós primários pedindo informações atualizadas sobre uso. Com base nas respostas, identifica o domínio menos sobrecarregado e com menor latência, e redireciona as novas requisições para ele.

% ----------------------------------------------------------
\subsection{Nós de Névoa}
% ----------------------------------------------------------

Os nós de névoa realizam o processamento distribuído e foram organizados em uma estrutura interna baseada em camadas, de forma a facilitar a manutenção e permitir a adição de novas funções sem comprometer o restante do sistema.

A primeira camada, denominada protocolos, é responsável por receber as requisições provenientes do nó primário, realizar as verificações e validações necessárias e repassar as informações para a camada seguinte. Essa estrutura é ilustrada na Figura~\ref{fig:camada_protocolos}, onde diferentes servidores de protocolos alimentam um protocolo padrão utilizado internamente.

\begin{figure}[htb]
    \caption{\label{fig:camada_protocolos}Camada de protocolos de um nó de névoa.}
    \begin{center}
        \includegraphics[width=0.7\linewidth]{images/camada_protocolos.png}
    \end{center}
    \fonte{Do autor.}
\end{figure}

A camada de processamento atua como núcleo de coordenação interna, gerenciando a comunicação entre as demais camadas e executando as necessidades específicas do nó de névoa. Essa camada também mantém um mecanismo para consultas internas de dados, permitindo que a camada de serviços acesse apenas as informações estritamente necessárias para seu funcionamento. Além disso, é por meio da camada de processamento que o nó de névoa realiza interações administrativas, como o registro inicial no nó primário. A Figura~\ref{fig:camada_processamento} mostra os principais componentes dessa camada.

\begin{figure}[htb]
    \caption{\label{fig:camada_processamento}Camada de processamento de um nó de névoa.}
    \begin{center}
        \includegraphics[width=0.7\linewidth]{images/camada_processamento.png}
    \end{center}
    \fonte{Do autor.}
\end{figure}

Na camada de serviços é executada a aplicação que implementa a lógica de pré-processamento dos dados. Utilizando a linguagem de consulta disponibilizada pela camada de processamento, essa camada obtém os dados necessários, executa o tratamento definido pela aplicação e devolve o resultado à camada de processamento, que, por sua vez, ajusta a saída para envio pela camada de protocolos.

% ----------------------------------------------------------
\subsection{Nó Agregador}
% ----------------------------------------------------------

O nó agregador recebe dados pré-processados dos nós de névoa e os consolida antes de enviá-los para a nuvem. Sua função é reunir informações de vários nós, realizando conversões de formato ou protocolo quando necessário para manter a compatibilidade com o sistema em nuvem.

Ao receber os dados, o agregador combina as informações em um único arquivo estruturado, que pode incluir médias, totais ou dados agrupados, dependendo do contexto da aplicação.

Esse processo reduz o tráfego de rede e o número de transmissões para a nuvem, melhorando o desempenho, economizando largura de banda e aumentando a eficiência. Após a consolidação, os dados são enviados para a camada de nuvem para análises mais detalhadas.

% ----------------------------------------------------------
\subsection{Nuvem}
% ----------------------------------------------------------

A camada de nuvem atua como o ponto central de processamento e gerenciamento de dados, lidando com tarefas que exigem mais poder computacional do que os nós de névoa podem oferecer.

Com uma infraestrutura robusta, ela recebe os dados pré-processados do nó agregador e executa a etapa final de análise e consolidação. Também realiza operações complexas, como aprendizado de máquina, inteligência artificial e análises preditivas, que demandam alto desempenho e grandes volumes de dados.

Além disso, a nuvem garante armazenamento seguro e permanente, mantendo dados históricos que apoiam a tomada de decisões, auditorias e estudos avançados.

Assim, a camada de nuvem complementa a arquitetura de névoa ao oferecer capacidade de processamento e armazenamento em larga escala.
