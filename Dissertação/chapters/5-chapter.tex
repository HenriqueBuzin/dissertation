% ----------------------------------------------------------
\chapter{Metodologia}
% ----------------------------------------------------------

\section{Tipo de Pesquisa}
Este trabalho caracteriza-se como uma pesquisa aplicada, de natureza tecnológica, com abordagem predominantemente qualitativa na fase de levantamento e definição de requisitos, e quantitativa na etapa de definição dos critérios de avaliação. O objetivo central é propor e modelar uma arquitetura modular de computação em névoa com suporte à comunicação entre múltiplas névoas.

\section{Procedimentos Metodológicos}

\subsection{Definição de Requisitos da Arquitetura}
A partir das lacunas identificadas na literatura (conforme discutido no Capítulo~\ref{cap:trabalhos_relacionados}), foram definidos requisitos funcionais e não funcionais, como:
\begin{itemize}
    \item Suporte à comunicação entre diferentes domínios de névoa;
    \item Capacidade de integração de dispositivos heterogêneos;
    \item Modularidade para implementação de múltiplos serviços;
    \item Balanceamento de carga e distribuição dinâmica de tarefas;
    \item Redução de latência e otimização do tráfego de rede.
\end{itemize}

\subsection{Modelagem da Arquitetura Modular}
A arquitetura proposta foi estruturada em três camadas principais:
\begin{itemize}
    \item \textbf{Nó Primário}: responsável pelo registro de dispositivos, balanceamento de carga entre nós locais e outros nós primários, e redirecionamento de tarefas.
    \item \textbf{Nós de Névoa}: responsáveis pela execução de serviços, pré-processamento e armazenamento temporário de dados.
    \item \textbf{Nó Agregador}: responsável por consolidar informações e preparar os dados para envio otimizado à nuvem.
\end{itemize}
Diagramas conceituais e funcionais foram elaborados para representar os componentes, fluxos de dados e protocolos de comunicação.

\subsection{Implementação em Ambiente Controlado}
A arquitetura foi implementada em ambiente local utilizando contêineres Docker, simulando medidores, nós de névoa e agregadores. Protocolos como HTTP e CoAP foram empregados para comunicação, e o gerenciamento de serviços foi desenvolvido em Python e Node.js. Essa configuração permitiu avaliar a viabilidade técnica e a escalabilidade horizontal da proposta.

\subsection{Definição dos Critérios de Avaliação}
Embora a validação experimental completa não tenha sido realizada no escopo deste trabalho, foram definidos critérios para avaliação futura:
\begin{itemize}
    \item Tempo de resposta (ms);
    \item Latência média (ms);
    \item Taxa de entrega de pacotes (\%);
    \item Escalabilidade (número de nós e dispositivos suportados).
\end{itemize}

\section{Ambiente e Ferramentas Utilizadas}
A implementação e simulação utilizaram os seguintes recursos:
\begin{itemize}
    \item Linguagens e frameworks: Python, Node.js;
    \item Infraestrutura de virtualização: Docker;
    \item Protocolos: HTTP, CoAP;
    \item Hardware: ambiente local com suporte a múltiplos contêineres e rede simulada;
    \item Ferramentas de modelagem: draw.io e ferramentas de diagramação compatíveis com \LaTeX.
\end{itemize}

\section{Fluxo Metodológico}
%A Figura~\ref{fig:fluxo_metodologia} apresenta o fluxo metodológico adotado, desde a definição dos requisitos até a implementação e definição dos critérios de avaliação.

%\begin{figure}[htb]
%	\caption{\label{fig:fluxo_metodologia}Fluxo metodológico do trabalho.}
%	\begin{center}
%		\includegraphics[width=0.9\linewidth]{images/fluxo_metodologia.png}
%	\end{center}
%	\fonte{Do autor.}
%\end{figure}
