% ----------------------------------------------------------
\chapter{Conclusão}\label{cap:conclusao}
% ----------------------------------------------------------

Este trabalho apresentou o desenvolvimento e a avaliação de uma arquitetura modular voltada a ambientes de computação em névoa, projetada para suportar serviços configuráveis e possibilitar a comunicação entre diferentes domínios de névoa. A solução proposta foi estruturada em camadas de protocolo, processamento e serviço, permitindo a integração de sistemas heterogêneos, a coordenação autônoma entre domínios e a distribuição eficiente de tarefas em ambientes distribuídos.

Os experimentos realizados com medidores de energia e água demonstraram que a arquitetura proposta mantém um fluxo de dados consistente desde a borda até a nuvem. A abstração de protocolos, aplicada entre HTTP e CoAP, simplificou a interoperabilidade entre dispositivos distintos, enquanto o uso da plataforma HPCC Systems na camada de nuvem possibilitou a agregação e a análise dos dados em larga escala.

Os resultados evidenciaram que duas instâncias de névoa puderam operar de forma coordenada e autônoma, realizando balanceamento dinâmico de carga. Além disso, a arquitetura apresentou baixo consumo de recursos, com aproximadamente 374~MiB no total, o que representa uma redução de cerca de 69\% em relação ao consumo relatado em implementações baseadas em Docker Compose. Esse comportamento reforça o caráter leve e eficiente da solução.

O sistema foi desenvolvido em conformidade com os principais princípios definidos pela \textit{OpenFog Reference Architecture (IEEE Std 1934-2018)}, especialmente quanto à escalabilidade, abertura, autonomia, programabilidade, e hierarquia. A coordenação entre nós primários implementa uma forma simplificada de tomada de decisão local, inspirada nos conceitos do OpenFog, contribuindo para o processamento distribuído e para a expansão flexível da infraestrutura \cite{openfog2018}.

De modo geral, os resultados confirmam que a arquitetura proposta alcança os principais objetivos da computação em névoa e os estende por meio da interoperabilidade entre domínios e do gerenciamento adaptativo de serviços.

Como perspectivas futuras, recomenda-se a integração de novos protocolos de comunicação, como MQTT, a adoção de mecanismos de autenticação e a criptografia ponta a ponta.
