% ----------------------------------------------------------
\chapter{Conclusão}\label{cap:conclusao}
% ----------------------------------------------------------

Este trabalho apresentou o desenvolvimento e a avaliação de uma arquitetura modular para ambientes de computação em névoa, com suporte à comunicação direta entre domínios distintos e integração entre dispositivos de borda, nós de névoa e a nuvem, tomando como referência o modelo proposto pelo OpenFog Consortium~\cite{openfog2018}. A solução foi projetada considerando os principais pilares dessa arquitetura no contexto de cidades inteligentes, buscando atender às diretrizes de escalabilidade, abertura, autonomia, programabilidade e hierarquia.

A escalabilidade foi contemplada na forma como a infraestrutura pode ser expandida sem reconfigurações complexas, permitindo a adição de novos nós ou dispositivos de maneira transparente. A abertura esteve presente no suporte a diferentes protocolos de comunicação, favorecendo a interoperabilidade entre sistemas heterogêneos. A autonomia foi explorada por meio do balanceamento de carga dinâmico e distribuído entre domínios, dispensando a necessidade de um ponto central de orquestração. A programabilidade se refletiu na flexibilidade para ajustar o comportamento do balanceador e dos nós de névoa de acordo com as necessidades do cenário de uso. Já a hierarquia esteve evidente na organização em camadas, interligando dispositivos de borda, névoa e nuvem de forma estruturada e modular.

Os resultados obtidos indicam que a combinação de abstração de protocolos, balanceamento dinâmico de carga e alto fator de agregação contribuiu para reduzir o volume de requisições à nuvem e melhorar a utilização dos recursos, mantendo tempos de resposta compatíveis com aplicações sensíveis à latência.

Na comparação com trabalhos como o EXEGESIS~\cite{exegesis2021}, que permite a substituição de serviços em ambiente de execução por meio da reconstrução do contêiner, a arquitetura proposta requer a interrupção para a troca de serviços. No entanto, a separação entre a camada de aplicação e as demais possibilita interromper apenas a aplicação, realizar os ajustes necessários e retomá-la sem modificar a lógica de comunicação. Durante esse processo, a camada de protocolos permanece ativa, continuando a receber dados, processá-los e armazená-los temporariamente, o que contribui para reduzir perdas e facilitar a retomada do serviço. Essa característica pode simplificar a manutenção e a adaptação a novos requisitos, minimizando o impacto de mudanças, ainda que a substituição implique uma breve indisponibilidade. Além disso, o encaminhamento distribuído entre névoas permite o tráfego direto de dados entre domínios, evitando a necessidade de um ponto único de orquestração.

Como contribuição, o trabalho demonstrou a viabilidade de integrar múltiplos protocolos e distribuir cargas entre diferentes domínios de névoa sem comprometer o desempenho. Para trabalhos futuros, sugere-se a adoção de mecanismos de autenticação e criptografia ponta a ponta, a implementação de estratégias de recuperação automática em caso de falhas e a aplicação da arquitetura em outros domínios, como saúde conectada e transporte inteligente, considerando diferentes perfis de tráfego e demanda.
